% !TeX spellcheck = be_NL
%---------- Inleiding ---------------------------------------------------------

\section{Introductie} % The \section*{} command stops section numbering
\label{sec:introductie}

In onze moderne maatschappij komen we vaak zowel direct als indirect in contact met allerlei soorten computers. Dit is niet zo abnormaal, aangezien onze samenleving is gefundeerd op de technologische ontwikkeling van de afgelopen paar jaar. Er is een onbetwistbare link tussen technologie en de toekomst. 
Logisch dat men steeds vaker beslist om kinderen meer in contact te brengen met technologie en computers. Zij zijn de toekomst van ons land. Zo trekt Ben Weyts\footnote{Vlaams minister van Onderwijs, sport, Dierenwelzijn en Vlaamse Rand} 375 miljoen euro uit voor digitale innovatie in het onderwijs \autocite{ArnoutGyssels2020}. 

De jeugd klaarstomen voor technologie is dus het doel.  Maar hoe weten de kinderen van de middelbare school hoe een computer werkt? Jongeren zijn immers nieuwsgierig naar hoe zoiets in elkaar zit. Er is dus een nood om, op een creatieve manier, jongeren aan te leren hoe een computer werkt en hoe je kan programmeren. 

Een mogelijke oplossing voor dit probleem, zijn zogenoemde “single board computers”. Deze kleine computers worden in verschillende omgevingen zoals spelcomputers, servers en zelfs weerstations gebruikt. Dit onderzoek vraagt zich af of we deze multifunctionele computer ook kunnen inzetten als leermateriaal in middelbare scholen. We zullen daarbij volgende onderzoeksvragen uitwerken:

\begin{itemize}
  \item Wat is een single board computer?
  \item Welke single board computers worden vandaag al ingezet in het onderwijs?
  \item Welke programmeertaal is optimaal voor het leren programmeren op een middelbaar onderwijs niveau?
  \item Welke voorbeelden van educatieve toepassingen bestaan al voor single board computers?
  \item Hoe voer je een educatieve toepassing van mini-computers uit?
  \item Wat zijn de voordelen van single board computers tegenover klassieke computers?
  \item Wat zijn de zwakke punten van single board computers?
  \item Wat zijn de plannen voor het gebruik van single board computers in het secundair onderwijs?
\end{itemize}

%---------- Stand van zaken ---------------------------------------------------

\section{State-of-the-art}
\label{sec:state-of-the-art}
\subsection{Wat is een single board computer}
Een single board computer is, zoals de naam impliceert, een computer waarvan alle essentiële onderdelen op een enkele printplaat staan.
Deze onderdelen voor het maken van een functionele computer bestaan uit het volgende:
\begin{itemize}
    \item Microprocessor
    \item Geheugen
    \item Input en ouput poorten
\end{itemize}

\subsection{Programmeertalen}
Deze mini-computers worden in het algemeen geprogrammeerd in Scratch\footnote{https://scratch.mit.edu/} en Python\footnote{https://www.python.org/}. Beide zijn makkelijk te begrijpen programeertalen, met Scratch dat zelfs speciaal is ontworpen voor kinderen. Men kan  ook andere programmeertalen gebruiken, zoals Java\footnote{https://www.java.com/nl/}
\autocite{Koelling2016}. 

\subsection{Waarvoor worden single board computers gebruikt?}
Door zijn goedkope aankoopprijs en “open source” eigenschappen, is een single board computer een aantrekkelijk apparaat om te gebruiken in verschillende projecten. Open source software betekent dat de broncode vrij beschikbaar is voor iedereen die ermee wil werken. De programeur moet dus geen auteursrechten of bijkomende kosten betalen bij het ontwerpen van nieuwe software. 

Hierdoor heb je veel variatie in de manier waarop men single board computers kan gebruiken. Zo kan het bijvoorbeeld gebruikt worden om uw huis te automatiseren \autocite{Jain2014}. Men kan ook simpelweg een website bouwen en deze hosten via één enkele single board computer. Je kan zelfs een zo'n mini-computer gebruiken om klassieke videogamekasten te maken\footnote{https://www.tech365.nl/retropie-lekker-retro-games-spelen-op-raspberry-pie/}. Er zijn ten slotte nog eindeloze andere mogelijkheden voor het gebruiken van een single board computer. Het is een universele technologie.

\subsection{Single board computers in het onderwijs}
Single board computers zijn ontworpen met educatie in gedachten. Zo zijn ze gemaakt om jongeren warm te maken voor programmeren en om interesse te kweken voor programmeer- en computerwetenschappen \autocite{Koelling2016}. Zoals eerder vermeld, is er zelfs een programmeertaal die gemaakt is voor kinderen: Scratch. 

Zo wordt in andere delen van de wereld al massaal ingezet in single board computers als leermateriaal. Deze mini-computers worden bijvoorbeeld gebruikt in Robot-kits: educationele projecten die kinderen helpen bij het leren van robotica \autocite{Junior2013}. 

In de middelbare scholen in New Mexico leert men, op hun eigen wiskundeniveau, programma's maken die in staat zijn tot foto- en videoherkenning. Dit doet men met behulp van een bepaalde single board computer: de Raspberry Pi\footnote{https://www.raspberrypi.org/}. De kinderen leren er programmeren in Python, een veel gebruikte programmeertaal van single board computers. Het project wordt altijd met veel plezier ontvangen bij de leerlingen. Dit toont het potentieel om op een creatieve manier te leren programmeren\autocite{MariosS.Pattichis2017}.

\subsection{Mogelijke meerwaarden}
Zoals eerder vermeld, zijn er veel opportuniteiten bij het introduceren van mini-computers in het secundair onderwijs. Door zijn goedkope aankoopprijs is het financieel aantrekkelijk voor scholen. Ook zijn mini-computers, zoals de naam impliceert, compact van formaat. Hierdoor kan het makkelijk opgeborgen en verplaatst worden.

Ook heeft het voordelen die specifiek zijn voor het secundair onderwijs. Mini-computers zijn ontworpen met educatie in gedachten. Hierdoor hebben de programmeertalen makkelijk te begrijpen en bezitten ze een vlakke leercurve. 

\subsection{Huidige hindernissen}
Er blijft wel een groot obstakel bij het introduceren van technologie in het onderwijs. Sommige middelbare scholen hebben de middelen noch het personeel om grote technologische projecten te organiseren. Ook blijft de kwestie van budget een knelpunt \autocite{Mouhamad2014}.
Gelukkig kan de introductie van single board computers hier bij helpen. Zoals eerder vermeld, zijn dit soort mini-computers relatief goedkoop. Ook zijn er online veel handleidingen die kunnen helpen bij leerkrachten. Zo zijn single board computers de ideale oplossing voor allerlei verschillende scholen.
%Hier beschrijf je de \emph{state-of-the-art} rondom je gekozen onderzoeksdomein. Dit kan bijvoorbeeld een literatuurstudie zijn. Je mag de titel van deze sectie ook aanpassen (literatuurstudie, stand van zaken, enz.). Zijn er al gelijkaardige onderzoeken gevoerd? Wat concluderen ze? Wat is het verschil met jouw onderzoek? Wat is de relevantie met jouw onderzoek?

%Verwijs bij elke introductie van een term of bewering over het domein naar de vakliteratuur, bijvoorbeeld~ \autocite{Doll1954}! Denk zeker goed na welke werken je refereert en waarom.

% Voor literatuurverwijzingen zijn er twee belangrijke commando's:
% \autocite{KEY} => (Auteur, jaartal) Gebruik dit als de naam van de auteur
%   geen onderdeel is van de zin.
% \textcite{KEY} => Auteur (jaartal)  Gebruik dit als de auteursnaam wel een
%   functie heeft in de zin (bv. ``Uit onderzoek door Doll & Hill (1954) bleek
%   ...'')

%Je mag gerust gebruik maken van subsecties in dit onderdeel.


%---------- Methodologie ------------------------------------------------------
\section{Methodologie}
\label{sec:methodologie}

%Hier beschrijf je hoe je van plan bent het onderzoek te voeren. Welke onderzoekstechniek ga je toepassen om elk van je onderzoeksvragen te beantwoorden? Gebruik je hiervoor experimenten, vragenlijsten, simulaties? Je beschrijft ook al welke tools je denkt hiervoor te gebruiken of te ontwikkelen.

In de startfase van dit onderzoek zal er bekeken worden wat de huidige situatie is van het gebruik van single board computers in het secundair onderwijs. Hierna zullen verschillende single board computers vergeleken worden om de meest optimale single board computer in het secundair onderwijs te bepalen. Na het vinden van de ideale mini-computer zal men zoeken naar de optimale programmeertaal. Ook hier zullen we een vergelijkende studie hanteren tussen verschillende programmeertalen. Beide vergelijkende studies zullen SWOT analyses bevatten. Dit onderdeel van de bachelorproef houdt het eerste deel van de literatuurstudie in.

Daarna zullen enkele praktische voorbeelden van educatieve toepassingen voor een single board computer uitgewerkt worden. Hiermee zal getest worden hoe moeilijk het is om zo'n project af te ronden en als dit makkelijk te organiseren is in een middelbare school omgeving. Ook bekijken we enkele obstakels die er eventueel zullen zijn bij het werken met single board computers.

Hierna zal theoretisch onderzocht worden als single board computers eventueel bestaande computers kunnen vervangen. Hiermee zal bekeken worden wat de voordelen en nadelen zijn van exclusief werken met single board computers.

In het laatste deel van dit onderzoek zal men bekijken als er plannen zijn om single board computers te integeren in de lessen op de middelbare school. Hiermee proberen we een beeld te scheppen over de toekomst en hoe single board computers daar eventueel een rol in kunnen spelen.

\subsection{Technologieën}

In dit onderzoek zal er gewerkt worden met een Raspberry Pi 3 Model B. Er zullen hiervoor eventueel wat externe hardware add-ons gebruikt worden (naargelang het gekozen project). Er zal vooral gewerkt worden met Python en Scratch. Eventueel kunnen andere frameworks gebruikt worden, afhankelijk van het gekozen project.
%---------- Verwachte resultaten ----------------------------------------------
\section{Verwachte resultaten}
\label{sec:verwachte_resultaten}
\subsection{Informatica op school}
Informatica leren op de middelbare school zal niks nieuw zijn. Er zullen al manieren zijn waarmee leerkrachten kinderen leren programmeren. Er zal echter nog weinig gebruik gemaakt worden van single board computers. Men kent de technologie nog niet of heeft weinig interesse in het implementeren van deze technologie in de lessen. Kinderen zullen ook weinig tot niks kennen van single board computers. Kinderen zullen waarschijnlijk enthousiast zijn. Leerkrachten zullen echter skeptisch en aarzelend zijn om single board computer te gebruiken in hun lessen.

\subsection{Educatieve voorbeeldprojecten}
De educatieve toepassingen voor single board computers zullen, dankzij hun kostprijs en hun speciaal ontworpen software, makkelijk te gebruiken zijn in de klas. De projecten zijn creatief en leuk om te maken. Er zullen genoeg documentaties zijn die leerkrachten kunnen ondersteunen in het organiseren van dit soort lessen. Er zal echter een bepaald niveau van taal- en wiskundekennis nodig zijn om deze projecten goed te kunnen afwerken. Single board computers zouden daarom perfect zijn om mee te werken in de eerste graad van het middelbaar onderwijs.

\subsection{Single board computers als substitutieproduct}
De verwachting is echter dat single board computers geen substitutie zullen zijn voor klassieke computer zoals desktops en laptops. De gelimiteerde hardware van single board computers zullen een bottleneck vormen voor grotere applicaties. Een bottleneck is een eigenschap van een product die het werken van andere onderdelen van dit product kan limiteren. Een single board computer kan dus geen normale computer vervangen \autocite{Kuss2018}.

\subsection{Toekomst van single board computers in het onderwijs}
Er zijn al plannen om meer informatica gerelateerde lessen te introduceren in het middelbaar onderwijs. Hiervoor zal men verschillende tools gebruiken om dit te realiseren. Het gebruik van single board computers zal hierin een rol spelen. 
%---------- Verwachte conclusies ----------------------------------------------
\section{Verwachte conclusies}
\label{sec:verwachte_conclusies}

Single board computers zullen een handige tool zijn om kinderen te leren hoe computers werken en om te leren programmeren. Dankzij hun compacte en draagbare design, hun gespecialiseerde software, hun lage aankoopkost en universele toepassingen zullen deze mini-computers vaker gebruikt worden in middelbare school als leermiddel.

