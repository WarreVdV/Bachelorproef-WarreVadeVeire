%%=============================================================================
%% Conclusie
%%=============================================================================

\chapter{Conclusie}
\label{ch:conclusie}

% TODO: Trek een duidelijke conclusie, in de vorm van een antwoord op de
% onderzoeksvra(a)g(en). Wat was jouw bijdrage aan het onderzoeksdomein en
% hoe biedt dit meerwaarde aan het vakgebied/doelgroep? 
% Reflecteer kritisch over het resultaat. In Engelse teksten wordt deze sectie
% ``Discussion'' genoemd. Had je deze uitkomst verwacht? Zijn er zaken die nog
% niet duidelijk zijn?
% Heeft het onderzoek geleid tot nieuwe vragen die uitnodigen tot verder 
%onderzoek?

Na het uitwerken van de lesvoorbereiding kunnen we antwoord geven op onze onderzoeksvragen. Zo kunnen we evalueren of deze bachelorproef daadwerkelijk geslaagd is of niet.

We hebben ontdekt dat de eindtermen voor programmeren verwerkt zitten in de einddoelen van 'Digitale competentie en mediawijsheid'. Om specifieker te zijn: de einddoelen rond computationeel denken. Zo moet een project de volgende bouwstenen bevatten:
\begin{itemize}
    \item Decompositie: Het opdelen van een groot probleem in kleinere, tussen problemen
    \item Abstractie: Het wegvallen van details om aan de essentie van een probleem te komen
    \item Patroonherkenning: Het benoemen en sorteren van soorgelijke problemen 
    \item Algoritmes: De tussen problemen opmsommen in een logische volgorde
\end{itemize} 
Deze eindtermen zijn noodzakelijk wanneer men in de eerste graad probeert een programmeerles op te stellen.

In het onderzoek hebben we meerdere voorbeelden gezien van single board computers die worden gebruikt in het onderwijs. Zo hebben we de 'Micro:bit': een single board computer die speciaal ontworpen is om te gebruiken in de les. Ook de Arduino: een microcontroller die eerst werd gebruikt voor doe-het-zelf-projecten, maar die een rol gekregen heeft in het leren programmeren. Tenslotte hebben we de Raspberry Pi in het breed besproken: Een single board computer die zowel voor het onderwijs als voor DIY-projecten kan gebruikt worden. We kozen uiteindelijk voor de Raspberry Pi, aangezien dit de meeste eigenschappen bevatte van een single board computer.

Hierna bekeken we verschillende programmeertalen voor de Raspberry Pi. Deze moesten makkelijk zijn om te begrijpen voor een eerste graad, maar ook complex genoeg zodat dit een echte eerste impressie ging geven over programmeren. Zo kozen we Scratch en Python als kandidaten. Beide hadden hun voordelen en nadelen, maar uiteindelijk ging de voorkeur naar Python. Deze makkelijk-te-gebruiken programmeertaal is namelijk een text-basedprogrammeertaal: de meest gebruikte vorm van programmeertalen in de professionele wereld, maar bekend voor zijn lage instap. 

Met de gekozen single board computer en programmeertaal, hebben we een aantal projecten overlopen. Deze waren allemaal waardige startprojecten, maar bij sommige was de complexiteit te hoog, terwijl bij andere de complexiteit te laag lag. Na wat zoeken hebben we uiteindelijk gekozen voor drie projecten:
\begin{itemize}
    \item een Hangspel
    \item Blad-steen-schaar
    \item een getal-raadspel
\end{itemize}
Op deze drie hebben we alle bouwstenen van computationeel denken toegepast en ontdekt dat ze uitermate kantidaten vormen. Later kozen we om het getal-raadspel uit te werken in een lesvoorbereiding. Dit document zou kunnen gebruikt worden door een leerkracht die interesse toont in het gebruik van een single board computer.

Deze lesvoorbereiding werd zelf uitgevoerd met behulp van een raspberry pi en een gewone computer. en werd doorverstuurd naar leerkrachten om hun feedback te verzamelen over het concept van een single board computer te gebruiken in de les:

De Raspberry Pi bleek in theorie een goede substitutie te vormen voor een laptop, maar in de praktijk ging dit wat stroever. De uitwerking van het project ging op beide harware even vlot: programmeren op beide apparaten gaf dezelfde ervaring. De Raspberry Pi heeft ook een intuïtieve besturingssyteem, waardoor de overstap van laptop naar single board computer vlot verliep. De gebruiksvriendelijkheid was echter een groot probleem op de Raspberry Pi. Zo werd er vaak haperingen vernomen doorheen het volledige gebruik van de PI en was de gebruikersinterface niet zo aantrekkelijk. Vooral de stroomtoevoer bleek een groot probleem te zijn. Zo kon de Pi op een willekeurig moment uitvallen, waardoor data verloren kon gaan. Tenslotte is het te afhankelijk van de randapparatuur (zoals beeldscherm en stroomkabel) om consistent een goede ervaring te krijgen.

Ook de keuze in programmeertaal leek te ambitieus. Volgens de feedback van de leerkrachten op de lesvoorbereiding, zou de syntax van Python een groot obstakel vormen. Ook met versimpelde termen, zoals een 'voorwaarde' in plaats van een 'if'-instructie, zou het niveau te moeilijk zijn voor leerlingen van de eerste graad. Door leerlingen te bombarderen met syntax-theorie, zou de concentratie op de essentiële doelstellingen van computationeel denken wat in het water vallen.

Na de resultaten verwerkt te hebben, kan men concluderen dat single board computers kunnen ingezet worden in het secundair onderwijs, maar met de nodige voorzichtigheid. Zo zou een uitwerking van een Python project in de tweede graad misschien beter zijn of zou het gebruik van een Micro:bit een betere gebruikerservaring geven. Ook zou het onderzoek misschien een beter resultaat geven als we gekozen hadden om het project te bouwen in Scratch. Deze vragen zouden kunnen uitgewerkt worden in een andere bachelorproef. Onze uitkomst lag dus ongeveer gelijk met onze verwachtingen, hoewel onze verwachtingen iets te optimistisch waren. We kunnen dus concluderen dat single board computers daadwerkelijk kunnen ingeschakeld worden, mits de juist voorbereiding en kennis.

