%%=============================================================================
%% Voorwoord
%%=============================================================================

\chapter*{\IfLanguageName{dutch}{Woord vooraf}{Preface}}
\label{ch:voorwoord}

%% TODO:
%% Het voorwoord is het enige deel van de bachelorproef waar je vanuit je
%% eigen standpunt (``ik-vorm'') mag schrijven. Je kan hier bv. motiveren
%% waarom jij het onderwerp wil bespreken.
%% Vergeet ook niet te bedanken wie je geholpen/gesteund/... heeft

Over de jaren heen heb ik een interesse ontwikkelt voor single board computers. Het idee van een volwaardige computer de grote van een creditkaart was iets wat ik vroeger moeilijk kon vatten. Ik zag voorbeelden van smart mirrors en retro spelcomputerkasten, allemaal dankzij single board computers. Jaren later kwam ik opnieuw in contact met single board computers dankzij het YouTube kanaal: Michael Reeves. Deze youtuber maakt gebruik van Raspberry Pi's en Arduino's om absurde robots te maken. In zijn video's verteld hij hoe makkelijk je robots kan maken met behulp van deze single board computers.  Zo maakte hij, tijdens het schrijven van dit onderzoek, een modificatie voor een robot hond van Boston Dynamics. Met deze modificatie kon de robot plots bier plassen. 

Geniaal en geweldig, maar het deed me denken: is programmeren op zo'n single board computer nu echt zo simpel? En zo ja: zouden kinderen dit kunnen gebruiken om te leren programmeren. Allebei mijn ouders zijn leerkracht, dus de connectie maken met onderwijs was makkelijk. Ik heb als kind vaak zelf moeten leren programmeren via YouTube. Een creatief medium zoals een single board computer leek mij ongelofelijk interessant, maar tot mijn verbazing werd dit nog niet zo vaak gebruikt in het huidige onderwijs. Daarom leek mij dit een leuke uitdaging om te bekijken hoe we single board computers kunnen introduceren bij leerkrachten.

Voor mijn onderzoek start, wil ik graag wat mensen bedanken die mij gesteund hebben tijdens het uitwerken:

Ik wil eerst en vooral mijn co-promotor, Annick van Daele bedanken. Haar interesse in mijn bachelorproef en haar zachtaardig karakter heeft mij altijd de motivatie gegeven om op volle toeren verder te werken. Zonder haar steun doorheen het project ging dit onderzoek niet half zo goed zijn geweest. De wereld van het onderwijs is relatief nieuw voor mij. Dus om een gids te hebben doorheen alle vaktermen en details, was ongelofelijk handig. Mevrouw van Daele heeft mij geholpen tot op het einde van mijn onderzoek; met het opzoeken van bronnen, uitleggen van vakjargon, het verbeteren van mijn taalfouten en doorwijzen naar leerkrachten voor mijn proof of concept. Ik ben zeker niet de makkelijkste om mee te werken, maar ze bleef enthousiast en hoopvol. Vanuit de grond van mijn hard, bedankt.

Hierna wil ik mijn promotor, Ludwig Stroobant, bedanken. Dankzij zijn duidelijke feedback en luisterend oor doorheen het project kan ik met veel trots dit document afwerken. In het begin wist ik niet goed waar te starten en wat precies te doen. Gelukkig stond meneer Stroobant altijd klaar om mijn vragen te beantwoorden op een kalme, duidelijke manier. Soms kon ik door de stress het bos door de bomen niet meer zien, maar dankzij meneer Stroobant lukte dit keer op keer opnieuw. Hartelijk bedankt!

Vervolgens moet ik een paar mensen bedanken die mij hebben geholpen doorheen mijn project. Zo wil ik Francis Wyffels bedanken voor zijn inbreng bij de keuze van single board computers. Dankzij zijn input kon ik de Micro:bit mooi klasseren in mijn onderzoek. Ook moet ik Bert van Vreckem bedanken. Dankzij corona had ik nog geen ervaring met LaTeX opgedaan in het tweede jaar. Ik ben hierdoor vaak vastgelopen met mijn bibliografie. Dankzij zijn hulp heb ik dit document mooi kunnen afwerken. Tenslotte wil ik nog de leerkachten bedanken die mij hebben geholpen bij mijn lesvoorbereiding, zeker Saen Callens en Lieve Van Bastelaere. Hun input maakt mijn onderzoek robuust. Ik wil hun bedanken voor hun tijd en interesse.

Tenslotte wil ik mijn vrienden en familie bedanken voor hun steun. Een bachelorproef maken is zeker geen flauwe kul. Gelukkig had ik altijd de support van mijn naasten om verder te werken. Vooral hebben mijn ouders mij vaak geholpen met vanalles en nog wat. Ze stonden altijd klaar voor mij. Ik zal dit nooit vergeten. Hartelijk bedankt!

  