%%=============================================================================
%% Samenvatting
%%=============================================================================

% TODO: De "abstract" of samenvatting is een kernachtige (~ 1 blz. voor een
% thesis) synthese van het document.
%
% Deze aspecten moeten zeker aan bod komen:
% - Context: waarom is dit werk belangrijk?
% - Nood: waarom moest dit onderzocht worden?
% - Taak: wat heb je precies gedaan?
% - Object: wat staat in dit document geschreven?
% - Resultaat: wat was het resultaat?
% - Conclusie: wat is/zijn de belangrijkste conclusie(s)?
% - Perspectief: blijven er nog vragen open die in de toekomst nog kunnen
%    onderzocht worden? Wat is een mogelijk vervolg voor jouw onderzoek?
%
% LET OP! Een samenvatting is GEEN voorwoord!

%%---------- Nederlandse samenvatting -----------------------------------------
%
% TODO: Als je je bachelorproef in het Engels schrijft, moet je eerst een
% Nederlandse samenvatting invoegen. Haal daarvoor onderstaande code uit
% commentaar.
% Wie zijn bachelorproef in het Nederlands schrijft, kan dit negeren, de inhoud
% wordt niet in het document ingevoegd.

\IfLanguageName{english}{%
\selectlanguage{dutch}
\chapter*{Samenvatting}
\lipsum[1-4]
\selectlanguage{english}
}{}

%%---------- Samenvatting -----------------------------------------------------
% De samenvatting in de hoofdtaal van het document

\chapter*{\IfLanguageName{dutch}{Samenvatting}{Abstract}}

In dit onderzoek wordt bekeken als single board computers kunnen gebruikt worden om leerlingen van de eerste graad secundair onderwijs te leren programmeren. In de moderne wereld zijn ICT-skills nog nooit zo noodzakelijk geweest voor elk kind. Veel kinderen worden geboren met een computer in hun schoot. We staan er niet meer bij stil hoe zo'n computer nu werkt. Tijd om daar verandering in te brengen. De Vlaamse overheid kiest er vaak voor om gewoon elk kind een laptop te geven, maar kan het niet anders? Zo'n investering is kostelijk, maar hoeft het zo te zijn?

Deze bachelorproef bekijkt als we single board computers kunnen gebruiken om leerlingen te leren programmeren. Zo onderzoekt het als we een project kunnen uitvoeren die alle eindtermen van de eerste graad secundair onderwijs aftikt, met behulp van een single board computer. De einddoelen voor digitale competentie en wijsheid worden bekeken en opgesomt. Hierna worden, dankzij SWOT-analyses, verschillende single board computer vergeleken en geëvalueerd. Hierna wordt slechts één uitgekozen. Verder zoeken we naar de ideale programmeertaal om te leren programmeren op een single board computer. Ook hierbij gebruiken we SWOT-analyses. Tot slot werken we een lesvoorbereiding uit voor een project die gebruik maakt van de gekozen single board computer en programmeertaal.

Uit het onderzoek vinden we dat de Raspberry Pi de beste kandidaat is voor een single board computer en Python de ideale programmeertaal. Hierna wordt een raadspel uitgewerkt met behulp van een lesvoorbereiding. Hierop wordt er feedback gegeven door leerkrachten van de eerste graad secundair.

We concluderen dat een single board computer kan gebruikt worden in het onderwijs, maar dat het niet de ideale oplossing is. Ook merken we dat Python toch niet de beste programmeertaal blijkt te zijn voor beginnende studenten. Het is dus belangrijk om duidelijk de juiste hardware en software te kiezen wanneer je gaat werken met single board computers. Tenslotte is de doelgroep nog te jong: een uitwerking in Python klinkt realistischer in een tweede graad.

Na het onderzoek vragen we ons af hoe single board computers kunnen evolueren om beter te werken in het onderwijs. Als uitbreiding van dit onderzoek zou er in de toekomst een les kunnen uitgewerkt worden door middel van Scratch in plaats van Python.
